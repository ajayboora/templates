\documentclass{beamer}
\usepackage{amsthm}
\usepackage{animate}

\title[An alternate title]{A presentation}
\author{Scott Sievert}
\institute{University of Wisconsin-Madison}
\date{April 29, 1992}

% Themes
\usetheme{Boadilla}
\usecolortheme{sidebartab}
\usefonttheme{structurebold} 
\setbeamertemplate{itemize items}[default]
\setbeamertemplate{enumerate items}[default]
\setbeameroption{show notes}

% change the footer
\makeatother
\setbeamertemplate{footline}{
    \leavevmode%
    \hbox{%
    \begin{beamercolorbox}[wd=.4\paperwidth,ht=2.25ex,dp=1ex,center]{author in head/foot}%
        \usebeamerfont{author in head/foot}\insertshortauthor
    \end{beamercolorbox}%
    \begin{beamercolorbox}[wd=.6\paperwidth,ht=2.25ex,dp=1ex,center]{title in head/foot}%
        \usebeamerfont{title in head/foot}\insertshorttitle\hspace*{3em}
        \insertframenumber{} / \inserttotalframenumber\hspace*{1ex}
    \end{beamercolorbox}}%
    \vskip0pt%
}
\makeatletter
\setbeamertemplate{navigation symbols}{}

% Fonts. Uncomment these if you want
%\usepackage{tgheros}
%\usepackage{tgadventor}
\usepackage{libertine}
%\usepackage[T1]{fontenc}

% For defining theorems
\newtheorem{mydef}{Golden ratio}

\begin{document}
\maketitle

\begin{frame}
    \frametitle{Animations}
    % PNG sequence generated with 
    % gifsicle --unoptimize figures/boot.gif | convert - figures/boot/boot-%d.png
    % HINTS:
    %   * remove --unoptimize flag if you get warning
    %   * mkdir figures/boot/ first
    % open with Adobe Reader to get this to work
    \animategraphics[loop,autoplay, width=\linewidth]{12}{figures/boot/boot-}{0}{30}
\end{frame}

\begin{frame}
    \frametitle{First slide}
    \begin{itemize}
        \item Bullets propery
        \item Quotes proper
        \item Thm boxes proper
        \item Animations
        \item Comments
    \end{itemize}
\end{frame}

\begin{frame}
    \frametitle{Quote}
    \begin{quotation}
        This is the best time!
    \end{quotation}
    I'm okay leaving this as is -- seems like a small deal.
\end{frame}

\begin{frame}
    \frametitle{Thm boxes}
    \begin{mydef}
        $x^2 + x + 1 = 0 \implies x = \phi \qed$
    \end{mydef}
\end{frame}

\end{document}
