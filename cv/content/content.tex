\vspace{-1.1em}

\texttt{\href{mailto:me@scottsievert.com}{me@scottsievert.com}}
$\circ$
\texttt{\href{\homepage}{blog}}
$\circ$
\texttt{\href{\github}{github}}

\vspace{-1em}

\SSsection{Goal}
    Earn a doctorate in signal processing and become an expert in the field.
    \vspace{-1em}

\SSsection{Research Interests}
    Inverse problems, adaptive sampling, learning theory, compressed sensing.

\SSsection{Education}
    \textbf{University of Wisconsin}, Madison, WI \hfill \textbf{Fall 2015 -- current}\\
    \textsl{M.S./Ph.D. in electrical and computer engineering}\\
    \textsl{Advisors: \href{http://willett.ece.wisc.edu}{Rebecca Willett} \& \href{http://nowak.ece.wisc.edu}{Robert Nowak}}
    \vspace{-0.5em}

    \textbf{University of Minnesota}, Minneapolis, MN \hfill \textbf{Fall 2010 -- Spring 2015}\\
    \textsl{Bachelors of Electrical Engineering (Honors, cum laude) } \hfill GPA: 3.729\\
    \textsl{Minor in Mathematics} \\
    \textsl{Senior Thesis/Project:} \href{\gesture}{Gesture}, an app to interpret sign language
    \vspace{-0.5em}

\SSsection{Undergraduate Research Experience}
    \SSsubsection{\href{\chasm}{Thermal camera} -- Research Assistant}{Advisor: Prof. Jarvis Haupt}{Summer 2013 -- Summer 2014}
    Typical thermal or infrared (IR) cameras cost between \$4,000 and \$40,000 dollars, the bulk of the cost going towards the IR sensor. Digital cameras, in essence, have a computer behind the sensor and thermal images are relatively simple. That prompts to us to develop an algorithm to minimize the number of sampled locations using a low-cost infrared sensor. Using a single sensor, Raspberry Pi, stepper motors and \href{\tree}{an algorithm} under my investigation as well as the Haar wavelet tree structure of natural images and the Fast Iterative Soft Thresholding Algorithm (FISTA) we can build a single-pixel thermal camera.
    Using this \$400 camera, we can deliver a reasonable image after sampling at about a 10\% rate.

    \SSsubsection{\href{\isparse}{iSparse} -- UROP}{Advisor: Prof. Jarvis Haupt}{Fall 2012 -- Spring 2013}
    An Undergraduate Research Opportunity (UROP) titled \href{\appReport}{``Compressive Sensing on the iPhone: Reconstructing Images from a Few Pixels.''} This resulted in an iPhone app called iSparse that uses the Fast Iterative Soft Thresholding Algorithm (FISTA) to reconstruct randomly sampled image by grouping the high frequency terms with the noise. Significant work was done to actually make this app usable by the public with reasonable performance, with the goal of creating a tool to inform the public and researchers in other disciplines about what is possible with signal processing. The final presentation was selected by University faculty to be used as an example of how UROP research should be presented. We are currently preparing an academic journal article and plan to release this app on Apple's Store.

\nocite{*}
\SSsection{Posters}
    \vspace{0.10em}
    \printbibliography[heading=none, type=inproceedings, resetnumbers=true]
\SSsection{Publications}
    \vspace{0.10em}
    \printbibliography[heading=none, type=unpublished, resetnumbers=true]


\SSsection{Honors and Awards}
    ECE Chancellor's Opportunities Fellowship, 2015-16
    \vspace{0.3em}\\
    University Honors Program, Spring 2011 -- graduation
    \vspace{0.3em}\\
    CSE Dean's List, September 2010 -- current (per 12 credit eligibility, minus 2013-14)
    \vspace{0.3em}\\
    Albert George Oswald Prize for outstanding research -- 2014-15
    \vspace{0.3em}\\
    College of Science and Engineering (CSE) Scholarship -- 2013-14, 2014-15
    %\vspace{0.3em}\\
    %Alpine Skiing State Qualifiers Academic Achievement Award, 2008--2010
    \vspace{0.3em}\\
    USCSA Ryan Smith Sportmanship Award, 2013
    \vspace{0.3em}\\
    Berggren Scholarship -- 2010

\SSsection{Activities}
    \href{https://cos.io}{Center for Open Science} ambassador,
    Honors Tutor (Calculus II),
    DRC faculty trainer

\SSsection{Selected Github repos}
\hspace{0.2em}
    \texttt{\href{\swix}{swix}},
    \texttt{\href{\xkcd}{xkcd-688}},
    \texttt{\href{\drawnow}{python-drawnow}},
    %\texttt{\href{\gesture}{gesture$^*$}},
    %\texttt{\href{\chasm}{chasm$^*$}},
    \texttt{\href{\isparse}{iSparse$^*$}}
    \hspace{0.4em}{\footnotesize{(\textrm{* private})}}


%%%%%%%%%%%%%%%%%%%%%%%%%%%%%%%%%%%%%%%%%%%%%%%%%%%%%%%%%%%%%%%%%%%%%%%%%%%%%%%%%%%%%
\pagebreak %%%%%%%%%%%%%%%%%%%%%%%%%%%%%%%%%%%%%%%%%%%%%%%%%%%%%%%%%%%%%%%%%%%%%%%%%%
%%%%%%%%%%%%%%%%%%%%%%%%%%%%%%%%%%%%%%%%%%%%%%%%%%%%%%%%%%%%%%%%%%%%%%%%%%%%%%%%%%%%%

\SSsection{Further Undergraduate Research}
    \SSsubsection{Research Assistant -- St. Anthony Falls Laboratory}{Advisor: Prof. Kimberly  Hill}{Summer 2011}
    I was responsible for the analysis of granular flow of sand and gravel using a high-speed camera. To do so, I modified an IDL image processing program and developed \href{\granFlowGithub}{a C++ application} that calculated the velocity profile of the flow by finding the flow direction, converting to real units, correcting for camera distortion, and eliminating erroneous data. The application also processed the velocity profiles to calculate relevant information. In addition, to reduce computation time, I deployed the IDL image processing application on up to 4 of Amazon's Elastic Cloud Compute (EC2) servers in parallel.

    \SSsubsection{UROP -- Electrical Engineering}{Mentor: Prof. William Robbins}{Fall 2011}
    A UROP  titled \href{\robbinsReport} {``Scaling a wind-driven, flag-like piezoelectric energy harvesting scheme."} I was asked to  find the optimal configuration of four flapping, piezoelectric elements in a wind tunnel while varying the sample's vertical separation, thickness, and length. The fundamental application of this technology would be to provide power in remote locations, using flapping flags to generate power (e.g., ocean bouys).

    \SSsubsection{UROP -- Civil Engineering}{Mentor: Prof. Kimberly Hill}{Spring 2011}
    A UROP titled \href{\hillReport}{``Particle segregation and flow in slurries: dependence on the interstitial fluid viscosity and angular velocity."} In mixtures of two different sized beads, special conditions (slow angular speed and viscous liquid) allow stripes to form. I investigated how changing those special conditions effected stripe width.


\SSsection{Student Groups}
\vspace{0.4em}
    \begin{tabularx}{\linewidth}{XX}
        \textbf{Alpine Ski Team} & \textbf{Winter 2011 -- Winter 2015}\\
            \textsl{\tab 2012 Treasurer}\\
            \textsl{\tab 2013, 2014 Co-vice president}\\
            \textsl{\tab 2014, 2015 Assistant Coach}\\
            \textsl{\tab 2015 Fundraising officer}\\

       \textbf{Wikipedia Racing Club} & \textbf{Fall 2011 -- Fall 2012} \\
            \textsl{\tab 2011--2012 President, Founder}\\

        \textbf{HKN} & \textbf{Spring 2013 -- Spring 2014} \vspace{0.2em}\\

        \textbf{Tesla Works} & \textbf{Fall 2011 -- Spring 2014}\\
        \tab \href{\rasterizeGithub}{\textsl{Giant Robopainter}} \textsl{Project Manager}\\

    \end{tabularx}

    \vspace{-0.7em}

\SSsection{Personal Side Projects}
    \textbf{Raspberry Pi Graphing Calculator}\\
        As a sophomore, I thought it would be useful to have a hand-held computer algebra system (CAS, such as \href{\sage}{Sage} or \href{\sympy}{SymPy}) while doing homework. To make such a device, I used a Raspberry Pi (RPi) and related components. After getting the RPi to run, I had to include all the parts in at least a  case and launch the CAS on startup and save the necessary battery power. This project was initiated with my own money for feasibility.  To complete the project, I wrote a grant for the UMN ECE Dept. Envision Funds for this project. The project was funded in March of 2013 and completed shortly thereafter. This project was featured in \href{http://www.ece.umn.edu/ECEENVISIONPROJECTS.html}{an issue of the ECE Alumni newsletter}.  %To be funded for an Envision Fund project as a sophomore is unusual.

\SSsection{Programming}
    Python, \LaTeX, Unix shell scripting, Matlab, Swift, C/Objective-C


\vfill

\SSsection{Selected Links}
    \textrm{Blog}\\
    \texttt{\url{\homepage}}

    \textrm{Github}\\
    \texttt{\url{\github}}

\vspace{6em}
